\section{Conclusions}

Our results suggest that KDT test design is complex with several levels of abstraction and that this design favours reusability; more than 60\% of the keywords are reused which has the potential of reducing the changes needed during evolution up to 70\%.

Additionally, we find that keywords change with a relatively low rate (approximately 5\%) indicating that after a keyword's creation only fine-grained, localised changes are performed by the testers. Our results suggest that the most common changes to KDT tests are caused by \emph{synchronization} or element \emph{location} changes between the SUT and the test suite and to the \emph{assertions} of the tests. Our findings indicate that during evolution 90\% of the keywords evolve and that test clones exist in KDT test suites; approximately 30\% of the keywords are duplicated. Finally, we report on the practitioners' perception on the challenges and benefits of adopting KDT.