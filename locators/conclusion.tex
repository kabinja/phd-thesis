\section{Conclusions}

We presented HPath, a novel \gls{dom}-based locator strategy to generate location paths more flexible for web testing. We have compared its potential to extract properties from the HTML document and its fragility under \gls{sut} evolution against state-of-the-art algorithm Robula+. Our results show that when \gls{html}5 semantics are present, HPath can exploit rendered properties of web elements to generate expressive locators in 73.35\% reducing locator breakages from 64.99\% when relying on attribute properties to 0.49\% with rendered properties. However, in its current form, it is not always able to extract rendered properties to create good location paths and leaks the hierarchical structure of the \gls{dom} for 41.48\% of the elements.

While HPath offers clear advantages in its expressiveness (only relying on rendered properties) and flexibility (when compared to Robula+), it suffers from some limitations. Indeed, relying on very specific predicates, the approach is not always able to generate short and expressive location paths, which might expose more of the internal hierarchy of the \gls{html} document. This effect is exacerbated when the targeted page is not relying on the current \gls{html}5 standards. However, despite this limitation, our results show that relying on the structure of the rendered \gls{dom} still remains better than exploiting the internal attributes of the elements from the \gls{html} document.