\section{Threats to Validity}

The goal of HPath is to ease interaction with visible components of a web page through automated GUI-based test scripts. Hence, HPath relies heavily on rendered properties to characterize target elements and only rendered elements are reachable by the approach. Nonetheless, section~\ref{sec:hpath-results-rq1} shows that elements that are not displayed (\emph{e.g.} containing $e_{div}$ of a modal window) can be targeted by the testers. This limitation shows that the goal of HPath is not to replace the currently available strategies but to offer better tool for automation engineers and answer their specific use case.

Furthermore, relying on the HTML standard, some elements provide information about the structure of the document but are not displayed. For instance, many elements from the Content Section category (see Table~\ref{tab:hpath-introduction-html5}) like $E_{article}$ or $E_{section}$ are not rendered but define the flow of the document. Though, because these elements do provide context and semantic, the current implementation of HPath keeps them to offer expressive locators that are easier to understand and potentially adapt.

Finally, conducting our case study on two projects, the conclusions we draw may not hold true for other projects. This constitutes a major threat to the external validity for the generalization of our results outside the context of this study. However, the two projects have very different profiles, thus making the study covering boundary cases among the full range of possible projects. To further alleviate this limitation, we provide a full replication package\footnote{Available at \emph{link not available for double blind process}} to encourage other teams to replicate and extend our results.

\section{Conclusions}

We presented HPath, a novel DOM-based locator strategy to generate location paths more flexible for web testing. We have compared its potential to extract properties from the HTML document and its fragility under SUT evolution against state-of-the-art algorithm Robula+. Our results show that when HTML5 semantics are present, HPath can exploit rendered properties of web elements to generate expressive locators in 73.35\% reducing locator breakages from 64.99\% when relying on attribute properties to 0.49\% with rendered properties. However, in its current form, it is not always able to extract rendered properties to create good location paths and  leaks the hierarchical structure of the DOM for 41.48\% of the elements.

Thus, in our short-term future work, we plan to (1) experiment HPath with more application by extending the support of Mercator to other languages implementing the Selenium API, (2) explore more opportunities to generate good predicates relying on heuristics present in the HTML documents such as section titles, overlays, etc. (3) explore ways to generate a rendering tree closer to the GUI components displayed on the screen to better abstract away structural details of the DOM.